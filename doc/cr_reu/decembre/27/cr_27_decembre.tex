\documentclass[french,a4paper]{article}
\setcounter{tocdepth}{4}
\setcounter{secnumdepth}{4}
\usepackage{booktabs}
\newcommand{\tabitem}{\textbullet~~}\title{Compte rendu de réunion}
\usepackage[bottom=2.5cm,top=2.5cm,left=2.5cm,right=2.5cm]{geometry}
\author{Noé Steiner - Alexis Marcel - Lucas Laurent - Mathias Aurand-Augier}
\date{27 Décembre 2022}
\begin{document}
\maketitle

\section*{\underline{Projet PPII - Compte rendu n°08 - réunion d'avancement}}

\begin{table}[!htb]
  \centering
  \begin{tabular}{| p{7cm} | p{7cm} |}
    \hline
    \multicolumn{1}{|c|}{ Motif / type de réunion:} & \multicolumn{1}{c|}{Lieu:} \\
    \hline
    \tabitem Alexis : Présent\newline
    \tabitem Noé : Présent\newline
    \tabitem Lucas : Présent\newline
    \tabitem Mathias : Présent                      &
    \tabitem Le 27 décembre 2022\newline
    \tabitem De 14h à 15h\newline
    \tabitem Visioconférence sur Discord                                         \\
    \hline
  \end{tabular}
\end{table}

\subsection*{\textit{Ordre du jour:}}

\begin{itemize}
  \item Répartition des tâches à la suite du développement du projet
  \item Progression developpement JSX
\end{itemize}

\subsubsection*{\textit{Informations échangées}}
\begin{itemize}
  \item Répartition des tâches pour le premier jalon de développement du projet :
    \begin{itemize}
      \item Alexis : modélisation jardin et gestion parcelles
      \item Lucas : Optimisation du code écrit
      \item Noé : modélisation jardin et gestion parcelles
      \item Mathias : Finaliser gestion de projet et commencer rapport
    \end{itemize}
\end{itemize}
\subsubsection*{\textit{Remarques / Questions}}
Que fera notre algorithme ?

\subsection*{\textit{Actions à suivre / Todo list}}
\begin{itemize}
  \item Vérification de la gestion du jardin
  \item Test du bon fonctionnement de l'application
\end{itemize}

\subsection*{\textit{Date de la prochaine réunion}}
La prochaine réunion aura lieu le samedi 7 janvier 2023, de 20h à 21h.

\end{document}