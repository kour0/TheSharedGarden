\documentclass[12pt,oneside,noprintercorrection]{iut}

%----------------------------------------------------------------------
%                     Chargement de quelques packages
%----------------------------------------------------------------------

% Si l'on produit le PDF avec pdflatex, ceci remplace la plupart
% des polices EC par des polices CM, plus adaptees a la generation de PDF,
% car ayant des equivalents PS :

\usepackage[cyr]{aeguill}

% pour les includegraphics
\usepackage{graphicx}

% mini "table of content"
\usepackage[french]{minitoc}
\usepackage[french]{babel}


% couleur des liens et hyperref -> mettre à {0.0,0.0,0.0} pour avoir du noir
%                               -> mettre à {0.2,0.2,0.2} pour avoir du gris foncé
\usepackage{color}
\definecolor{linkcolor}{rgb}{0.1,0.1,0.7}
\usepackage[hypertexnames=false]{hyperref}
\hypersetup{
    colorlinks,%
    citecolor=linkcolor,%
    filecolor=linkcolor,%
    linkcolor=linkcolor,%
    urlcolor=linkcolor,%
}

% Pour les codes
\usepackage{listings}
\lstset{language=C++,basicstyle=\small}


\usepackage{silence}

\WarningFilter{minitoc(hints)}{W0023}
\WarningFilter{minitoc(hints)}{W0024}
\WarningFilter{minitoc(hints)}{W0028}
\WarningFilter{minitoc(hints)}{W0030}
\WarningFilter{hyperref}{bookmark level}

\WarningFilter{blindtext}{} % this takes care of the `blindtext` messages

%-------------------------------------------------------------------
%  Surcharge de commandes pour les variables et page d'en-tête
%-------------------------------------------------------------------

\makeatletter

%
% les deux commandes suivantes sont entre \makeatletter
% et \makeatother parce qu'elles utilisent des `@'.
%

\renewcommand{\@DFD}{Universit\'e de Lille\\ IUT de Lille\\ D\'epartement informatique}


\renewcommand{\@Lillehe@d}{{\UseEntryFont{ThesisFirstPageHead}\noindent
    \centerline{\if@logo@uhp@
                    {\setbox0=\hbox{$\raise2.3cm\hbox{\UHPLogo}$}%
                     \ht0=\baselineskip\box0}\hfill
                \else
                    Universit\'e Lille%
                \fi}%
    \@TL@cmn@head\\
    \par
    }%
    }


\newcommand\TheseLilleI{\renewcommand{\@ThesisFirstPageHead}{\@Lillehe@d}%
                         \ThesisDiploma{{\UseEntryFont{ThesisDiploma}%
                              \\[3mm]
            {\UseEntryFont{ThesisSpecialty}( )}}}}

\makeatother

%-------------------------------------------------------------------
%           Corrections pour les imprimantes recto-verso
%                          (A AJUSTER)
%-------------------------------------------------------------------

%\ShiftOddPagesRight{-1mm}
%\ShiftOddPagesDown{2.5mm}
%\ShiftEvenPagesRight{0mm}
%\ShiftEvenPagesDown{0mm}

%-------------------------------------------------------------------
%                Mise en page
%-------------------------------------------------------------------

%-------------------------------------------------------------------
%                             interligne
%-------------------------------------------------------------------
\renewcommand{\baselinestretch}{1.3}

%-------------------------------------------------------------------
%                             Marges
%-------------------------------------------------------------------

% pour positionner les vraies marges:
%\SetRealMargins{1mm}{1mm}

%-------------------------------------------------------------------
%                             En-tetes
%-------------------------------------------------------------------
%On n'utilise pas les logos
%\DontShowLogos

% Les en-tetes: quelques exemples
%\UppercaseHeadings
%\UnderlineHeadings
%\newcommand\bfheadings[1]{{\bf #1}}
%\FormatHeadingsWith{\bfheadings}
%\FormatHeadingsWith{\uppercase}
%\FormatHeadingsWith{\underline}
\newcommand\upun[1]{\uppercase{\underline{\underline{#1}}}}
\FormatHeadingsWith\upun

\newcommand\itheadings[1]{\textit{#1}}
\FormatHeadingsWith{\itheadings}

% pour avoir un trait sous l'en-tete:
\setlength{\HeadRuleWidth}{0.4pt}


%-------------------------------------------------------------------
%                         Les references
%-------------------------------------------------------------------

\NoChapterNumberInRef \NoChapterPrefix

%-------------------------------------------------------------------
%                           Brouillons
%-------------------------------------------------------------------

% ceci ajoute une marque `brouillon' et la date
%\ThesisDraft




\renewcommand{\labelitemi}{$\bullet$}
\renewcommand{\labelitemii}{$\circ$}
%-------------------------------------------------------------------
%                          Encadrements
%-------------------------------------------------------------------

% encadre les chapitres dans la table des matieres:
% (ces commandes doivent figurer apres \begin{document}

%\FrameChaptersInToc
%\FramePartsInToc


%-------------------------------------------------------------------
%            Reinitialisation de la numerotation des chapitres
%-------------------------------------------------------------------

% Si la commande suivante est presente,
% elle doit figurer APRES \begin{document}
% et avant la premiere commande \part
\ResetChaptersAtParts

%-------------------------------------------------------------------
%               mini-tables des matieres par chapitre
%-------------------------------------------------------------------

% preparer les mini-tables des matieres par chapitre.
% (commande de minitoc.sty)
%\dominitoc

\NewJuryCategory{EncadrantEts}{\it Encadrant entreprise :}{\it Encadrant entreprise:} 
\NewJuryCategory{EncadrantUniv}{\it Encadrant universitaire :}{\it Encadrant universitaire:} 

\TheseLilleI





%-------------------------------------------------------------------
%                         Page de titre:
%-------------------------------------------------------------------


\ThesisTitle{Titre du stage}
\ThesisKind{Rapport de stage}
\ThesisPresentedThe{soutenu le XXXX}
\ThesisAuthor{Prénom NOM}

\NomDuLaboOuEntreprise{Nom de l'entreprise}
\LogoLaboOuEntreprise{img/blankImage} % Image du logo du labo ou ets dans le rep img

\EncadrantEts = {XXX}
\EncadrantUniv = {YYY}

\begin{document}

% Creation de la page de titre:
\MakeThesisTitlePage



%-------------------------------------------------------------------


%-------------------------------------------------------------------
%                          remerciements
%-------------------------------------------------------------------

%\DontFrameThisInToc
\begin{ThesisAcknowledgments}

\end{ThesisAcknowledgments}





%-------------------------------------------------------------------
%                  ecriture de `Chapitre' et `Partie'
%                      dans la table des matieres
%-------------------------------------------------------------------

\WritePartLabelInToc \WriteChapterLabelInToc


%-------------------------------------------------------------------
%                        table des matieres
%-------------------------------------------------------------------

\tableofcontents

%-------------------------------------------------------------------
%              Exemple d'utilisation de \SpecialSection
%-------------------------------------------------------------------

% La commande \mainmatter (nouvelle commande LaTeX2e) permet de passer
% a la numerotation arabe (ce que fait \pagenumbering{arabic})
% et de faire commencer la nouvelle page 1 sur une page impaire.
% On evitera donc d'utiliser directement \pagenumbering{arabic}.
\mainmatter

% ----------------------------------------------------------------
\SpecialSection{Résumés}

Résumés en français et en anglais du rapport.

\SpecialSection{Introduction}



% Pour ne pas avoir le mot `Chapitre' au debut de chaque chapitre.
\NoChapterHead


%--------------------------------------------------------------------------------------
%--------------------------------------------------------------------------------------
\chapter{Prise en main de \LaTeX}

Il est possible d'utiliser \LaTeX directement dans un navigateur avec des outils comme OverLeaf mais vous pouvez aussi compiler le code \LaTeX directement sur votre machine.

\section{Installation}

\subsection{Sous Linux}
Nombreuses documentations disponibles sur internet pour
l'installation des packages. Sous {\sc Ubuntu} par exemple, 
le package {\tt texlive} installe une sélection des outils les 
plus fréquements utilisés.

\subsection{Sous Windows}

\begin{itemize}
\item Pour compiler les fichiers {\tt .tex} en {\tt .pdf}, installer  {\sc Miktex} \cite{Miktex}
\item Pour écrire des documents \LaTeX, installer  {\sc TeXnicenter} \cite{Texnicenter} ou  {\sc ConTEXT} \cite{ConTEXT}.
\end{itemize}

\subsection{Sous Mac}
Utiliser par exemple MacTeX. 

\subsection{Figures}
\label{subsection:figures}

Pour créer vos propres figures, vous pouvez utiliser {\sc Inkscape} \cite{Inkscape} (ou éventuellement l'outil {\sc Draw} d'OpenOffice \cite{OpenOffice}) pour réaliser des dessins vectoriels; il est également possible d'utiliser {\sc Gimp} \cite{Gimp} pour réaliser des dessins bitmaps.

Notez qu'{\sc Inkscape} sait réaliser la plupart des conversions vectorielles vers {\tt pdf}, et que {\sc Gimp} sait réaliser la plupart des conversions bitmaps vers {\tt png} ($\rightarrow$ schémas) ou {\tt jpg} ($\rightarrow$ photos).

\section{Compilation des documents}
Pour compiler un document \LaTeX  en pdf, le plus simple est d'utiliser la commande {\tt pdflatex}; 
il est pour cela nécessaire d'inclure les images dans les formats {\tt .pdf}, {\tt .jpg}, ou {\tt .png}
comme indiqué en section \ref{subsection:figures}.


Par exemple, pour compiler ce document, les commandes suivantes ont été lancées :
\begin{center}
  \fbox{
    {\scriptsize\tt
      pdflatex rapport~;~~ bibtex   rapport~;~~pdflatex rapport~;~~pdflatex rapport
    }
  }
\end{center}            
\section{Quelques commandes}
\subsection{Insertion de figures}

\begin{figure}[!ht]
  \centering
  \includegraphics[width=8cm]{img/UL}
  \caption{Le logo de l'Université de Lille.}
  \label{fig:univ-lille}
\end{figure}

Voici le Logo de l'Université de Lille (voir fig. \ref{fig:univ-lille}) en vectoriel et en gros ... Vous pouvez mettre les images dans le répertoire \texttt{img} si vous modifiez ce fichier exemple ou en ajoutez d'autres selon ce modèle.

\subsection{Insertion d'équations}

Les équations (et autres formules) sont un des points forts de \LaTeX, utile si vous devez formaliser votre travail. Quelques exemples ci-dessous\ldots

Pour une fonction $f(x)$ continue et croissante sur l'intervalle $[a..b]$, l'équation \ref{eq:sumint} sert à \ldots

\begin{equation}
\sum_{i=a}^{b-1} f(i)   \leq   \int_{a}^{b}f(t)\;dt   \leq   \sum_{i=a+1}^{b} f(i)
\label{eq:sumint}
\end{equation}

ou alors la matrice Vandermonde \ref{eq:matvan} sert à \ldots

\begin{equation}
V=
\left( 
\begin{array}{ccccc}
1 & \alpha_1 & \alpha_1^2 & \dots & \alpha_1^{n-1}\\
1 & \alpha_2 & \alpha_2^2 & \dots & \alpha_2^{n-1}\\
1 & \alpha_3 & \alpha_3^2 & \dots & \alpha_3^{n-1}\\
\vdots & \vdots & \vdots & \ddots &\vdots \\
1 & \alpha_m & \alpha_m^2 & \dots & \alpha_m^{n-1}\\
\end{array}
\right) 
\label{eq:matvan}
\end{equation}

ou encore un définition récursive \ref{eq:fac} qui peut servir dans certains cas à \ldots

\begin{equation} 
  fact(n) = \left\{ 
    \begin{array}{ll}
      1                  & \mbox{si $x \leq 1$};\\
      n \times fact(n-1) & \mbox{autrement}.
    \end{array} 
  \right.
  \label{eq:fac}
\end{equation}


Un document qui peut vous être utile est le suivant \cite{mshort}. La documentation {\em The Not So Short Introduction to \LaTeX} \cite{lshort} présente également des examples mathématique assez détaillés.
\newpage % saut de page pour éviter certains mauvais formatages

\subsection{Insertion de code}

Pour plus d'infos sur le package {\tt listings}, consulter cette note de bas de page\footnote{
\url{ftp://tug.ctan.org/pub/tex-archive/macros/latex/contrib/listings/listings.pdf}

}

\begin{lstlisting}[frame=trBL]
#include <iostream>

int main() {
    std::cout << "Hello, world!\n";
}
\end{lstlisting}

\section{Doc \LaTeX}
Faire des recherches sur Google ou consulter ce livre très complet \cite{LatexCompanion}. La documentation {\em The Not So Short Introduction to \LaTeX} \cite{lshort} est également un très bon point de départ; elle est disponible en ligne.

\SpecialSection{Conclusion}


\bibliographystyle{myunsrt}
\small
\bibliography{rapport}

\Annex{Annexe 1}

\end{document}
