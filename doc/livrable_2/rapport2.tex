\documentclass[french,a4paper]{article}
\setcounter{tocdepth}{4}
\setcounter{secnumdepth}{4}
\usepackage{float}
\usepackage{graphicx}
\usepackage{hyperref}
\usepackage{pdfpages}
\newcommand{\tabitem}{\textbullet~~}
\newcommand{\HRule}{\rule{\linewidth}{0.5mm}}
\usepackage{multirow}
\graphicspath{{img/}}
\title{PPII}
\usepackage[bottom=2.5cm,top=2.5cm,left=2.5cm,right=2.5cm]{geometry}
\author{Noé Steiner - Alexis Marcel - Lucas Laurent - Mathias Aurand-Augier}
\date{Janvier 2023}
\begin{document}

%\maketitle

\begin{titlepage}
    \begin{center}

    \includegraphics[width=0.5\textwidth]{tele_univ.png}

    \textsc{\Large Rapport final de Projet Pluridisciplinaire d'Informatique Intégrative}\\[1.5cm]

    \HRule \\[0.4cm]
    { \huge \bfseries Les jardins partagés\\[0.4cm] }

    \HRule \\[2cm]

    \begin{minipage}{0.4\textwidth}
      \begin{flushleft} \large
        Alexis MARCEL\\
        Lucas LAURENT\\
        Noé STEINER\\
        Mathias AURAND-AUGIER\\
      \end{flushleft}
    \end{minipage}
    \begin{minipage}{0.4\textwidth}
      \begin{flushright} \large
        \emph{Responsable du module :}\\
        Olivier FESTOR\\
        Anne-Claire HEURTEL\\
        Gerald OSTER\\
      \end{flushright}
    \end{minipage}

    \vfill

    {\large 6 Janvier 2023}

  \end{center}
\end{titlepage}
\newpage
\tableofcontents
\newpage
\section{Base de donnée}
\subsection{Conception}
\subsubsection{Les besoins de notre application}
Pour commencer, notre application permet à des utilisateurs de s'enregistrer sur la plateforme. Une compte utilisateur est composé d'un 
email, un pseudo, son prénom, son nom, ainsi que la date de la création de son compte. Ensuite un utilisateur peut créer un jardin
et ce jardin peut être rejoint par d'autres utilisateurs.
Un jardin est constitué d'un nom, un propriétaire, un type (public ou privé) et enfin une adresse. Le propriétaire du jardin peut 
ensuite créer des parcelles dans son jardin. Une parcelle est composée d'un nom et d'une plante. Une plante à un nom et un besoin en eau. Les parcelles sont elle même consituées
d'unités qui réprésentent un espace réel dans le jardin. Une unité est caractérisé par un index déterminant sa position dans le jardin.
De plus, chaque parcelle peuvent être associé à une liste de tâche à effectuer. Une tâche est caractérisé par une date limite, un nom, 
une description, un état (à faire, en cours, terminé), une date limite et enfin une personne qui s'en occupe.

\subsubsection{Schéma entité-association}

Ces besoins donnent lieu à la création de plusieurs entités consituant le schéma entité-association suivant :

\begin{figure}[H]
    \centering
    \includegraphics[width=1\textwidth]{img/Schema_entite_association_PPIIversion2.drawio.png}
    \caption{Schéma entité-association}
\end{figure}

Ce schéma respecte les contraintes logiques de cardinalités suivantes : 

\begin{itemize}
    \item Une personnes peut posséder/rejoindre un ou plusieurs jardins.
    \item Un jardin peut avoir plusieurs parcelles mais une parcelle est associée à un seul jardin.
    \item Une parcelle peut avoir plusieurs unités mais une unité est associée à une seule parcelle.
    \item Une parcelle peut avoir une plante et une plante peut être attribué à plusieurs parcelles.
    \item Une personne peut effectuer plusieurs tâches et une tâche peut être effectuée par plusieurs personnes.
\end{itemize}

\subsubsection{Passage du modèle au entité-association au relationnel}
A présent, nous allons transformer notre schéma entité-association en modèle relationnel en respectant les règles de la troisième forme
normale.

\begin{itemize}
    \item account(id, email, username, first name, last name, password, created at)
    \item garden(id\_garden, garden\_name, owner, manager, garden\_type, street\_adress, country, city, province, postal\_code)
    \item plot(plot\_id, garden\_id, plot\_state, plot\_name, plant)
    \item task(task\_id, plot\_id, task\_name, task\_description, task\_manager, task\_state, completion\_state, validation\_state, deadline)
    \item plot\_unit(plot\_id, unit)
    \item plant(id, plant\_name, water\_need)
    \item do(account\_id, task\_id)
    \item link(account\_id, garden\_id)
\end{itemize}
\subsection{Implémentation de la base de donnée dans le backend de l'application}

\subsubsection{Création de la base de données}
En utilisant le systeme de gestion de base de données sqlite, nous avons créé notre base dans un fichier data.db avec le script 
SQL sauvegardé dans un fichier nommé "creation table.sql". Ces deux fichiers sont dans le dossier backend/data du projet. 


\subsubsection{Utilisation de SQLAlchemy}
SQLAlchemy est un ORM (Object-Relational Mapping) permettant de manipuler la base de données via des objets python. Les requêtes en 
python sont ainsi "traduites" en SQL et la réponse reçu se présentera sous la forme d’un objet python avec lequel on peut interagir.
SQLAlchemy constitue donc un pont entre la base de données et notre application. Pour que l’ORM puisse fonctionner, il faut définir
des classes python qui correspondent aux tables de la base de données. Ces classes ont des attributs qui correspondent aux attributs 
des tables. On appelle ça des modèles. Les modèles sont ensuite utilisés pour créer des requêtes SQL.

Les sessions de SQLAlchemy permettent de gérer les transactions SQL, autrement dit un ensemble de requêtes. Si l'une d'elles échoue,
l'ensemble de la transaction est annulée et aucune requête n'est communiquée à la base. L’avantage de ce système est la sécurité.

\newpage
\section{Serveur et client Web}
\subsection{structure de l'application}
L'application est divisée en deux parties afin de pouvoir maitriser complètement le coté client, important pour la modélisation de jardins :
\begin{itemize}
    \item Le backend, qui est le coeur de l'application. C'est un serveur web flask. Celui-ci est décomposé en une multitudes de routes, retournant toutes du JSON. Il s'agit d'une API REST.
    \item Le frontend, qui est la partie visible de l'application. Il est entièrement réalisé avec Javascript, accompagné de la librairie React.js. Le frontend communique avec le backend via des requêtes HTTP, réalisée à l'aide de la librairie Axios.
\end{itemize}
\subsection{Fonctionnalités de l'application}
\subsubsection{Authentification}
L'application dispose d'un système d'authentification complet, afin de sécuriser l'ensemble, ainsi que pour personnaliser les fonctionnalités des utilisateur. Cette authentification se déroule via des tokens JWT, générées par le serveur au moment de la connection, puis stockés dans les cookies du navigateur de l'utilisateur. Ces tokens sont ensuite utilisés pour vérifier l'identité de l'utilisateur, et ainsi lui permettre d'accéder aux routes protégées.
Cette authentification permet entre autres à l'utilisateur de créer, rejoindre des jardins, de les gérer, et de les partager en leur nom.

\subsubsection{Gestion de compte}
Un utilisateur authentifié dispose de plusieurs fonctionnalités afin de personnaliser et modifier son compte. Il peut ainsi modifier ses informations personnelles, telles que sa photo de profil et son nom.

\subsubsection{Gestion des jardins}
De nombreuses routes sont dédiées au management des jardins sur l'application. On peut en effet retrouver ces différentes opérations :
\begin{itemize}
    \item Récupérer un jardin, en fonction de son identifiant.
    \item Créer un jardin, en lui donnant un nom, une description, une adresse et en choisissant le type de jardin (privé ou public).
    \item Supprimer un jardin qu'un utilisateur a créé.
    \item Modifier les informations et les membres d'un jardin.
    \item Rejoindre un jardin.
\end{itemize} 

\subsubsection{Carte interractive}
Afin de rendre plus accessible la fonctionnalité de rejoindre un jardin, il est également possible de le faire depuis une carte interractive, sur laquelle est listée l'intégralité des jardins publiques, ainsi que ceux dont l'utilisateur est membre. Cette carte est réalisée à l'aide de la librairie Leaflet.

\subsection{Interractions avec le jardin}
Les utilisateurs membres de jardins peuvent interagir avec celui-ci, en fonction de leurs permissions. On peut ainsi retrouver les fonctionnalités suivantes :
\begin{itemize}
    \item Créateur d'un jardin : 
    \begin{itemize}
        \item Modeliser le jardin
        \item Gérer les utilisateurs du jardin
        \item Supprimer le jardin
    \end{itemize}
    \item Les membres et le créateur :
    \begin{itemize}
        \item Ajouter / supprimer des tâches à une parcelle
        \item Modifier les plantes des parcelles
    \end{itemize}
\end{itemize}

\newpage
\section{Algorithme}
\subsection{Principe de l'algorithme}
Le problème à résoudre ici est de pouvoir alimenter en eau le potager de manière optimale. Pour cela, nous sommes parti sur l'idée de placer des bouteilles d'eau goutte à goutte sur les parcelles du potager. 
Cela permet de pouvoir arroser les plantes de manière optimale, en fonction de leurs besoins.
Mais on souhaite placer de manière optimale les bouteilles d'eau sur les parcelles pour en placer le moins possible.
\subsection{Implémentation}
Tout d'abord, nous avons d'abord modéliser le jardin sous la forme d'une matrice avec le module numpy pour avoir une matrice optimisée en python.
Ensuite, nous avons implémenté l'algorithme de placement des bouteilles d'eau. Pour cela, on a posé des contraintes :
\begin{itemize}
    \item Une bouteille d'eau alimente en eau la parcelle sur laquelle elle est placée et les parcelles adjacentes.
    \item Une parcelle porte un poids qui correspond à la quantité d'eau dont elle a besoin.
    \item Une bouteille d'eau ne peut être placée que sur une parcelle.
    \item Une bouteille d'eau n'arrose qu'une fois une parcelle, ainsi une plante ayant un besoin de deux devra être arrosée par deux points d'eau différents et ainsi de suite.
\end{itemize}
Notre algorithme utilise Gurobi, une bibliothèque, 
\subsection{Analyse en complexité}
\subsection{Test de validité de l'algorithme}
\subsection{Analyse de performance}

\newpage
\section{Gestion de projet}
\subsection{Équipe de projet}
Ce projet est un projet local réalisé en groupe de 4 personnes~:
\begin{itemize}
    \item Alexis MARCEL
    \item Lucas LAURENT
    \item Noé STEINER
    \item Mathias AURAND-AUGIER
\end{itemize}
Le comité de pilotage est constitué de~:
\begin{itemize}
    \item Anne-Claire HEURTEL
    \item Olivier FESTOR
    \item Gérald OSTER
\end{itemize}
Ces personnes constituent les parties prenantes de notre projet ainsi que les acteurs influents sur le livrables.
\begin{figure}[H]
    \centering
    \includegraphics[width=0.75\textwidth]{img/parties_prenantes.png}
    \caption{Parties prenantes}
\end{figure} 
\subsection{Organisation au sein de l’équipe projet}
Nous avons réalisé plusieurs réunions, en présentiel dans les locaux de Télécom Nancy mais également sur en visio-conférence sur Discord. Ces réunions nous ont permis de mettre en commun nos avancés régulièrement, de partager nos connaissances sur des problématiques et de nous organiser de manière optimale.
Les comptes rendus des réunions réalisés sont présents dans l’\hyperlink{annexe1}{Annexe 1}.

De plus, dès le début de notre projet nous avons mis en place un projet Trello. Trello est une application permettant d’organiser facilement un projet en reposant sur une organisation en planches listant des cartes, chacune représentant des tâches. Ces tâches peuvent ensuite être déplacées permettant de découper notre projet en plusieurs jalons dynamiquement.
\begin{figure}[H]
    \centering
    \includegraphics[width=0.75\textwidth]{img/trello.png}
    \caption{Organisation Trello}
\end{figure} 

Ensuite, nous avons utilisé GitLab pour gérer les différentes versions du développement de notre application, ainsi que les différentes branches nous permettant de travailler simultanément sans conflit.

Enfin, la rédaction des differents comptes rendu de réunion et des rapports ont été rédigé en \LaTeX.

\subsection{Objectifs SMART}
La méthode SMART que l'on rappelle ci-dessous nous a permis de définir nos différents objectifs :

\begin{figure}[H]
    \centering
    \includegraphics[width=1\textwidth]{img/SMART.png}
    \caption{Objectif SMART}
\end{figure}

\subsection{Matrice des objectifs}
Nous avons conçu, à l'aide de la méthode SMART, la matrice des objectifs suivante :

\begin{figure}[H]
    \centering
    \includegraphics[width=1\textwidth]{img/matrice_des_objectifs.png}
    \caption{Matrice des objectifs}
\end{figure}

\subsection{Triangle qualité-cout-délai}
Afin d’établir des objectifs cohérents, et réalisables dans les délais, nous avons réalisé le triangle qualité-coût-délai. On remarque ainsi, les délais étant courts, que nous avons tout intérêt à ne pas se fixer des objectifs trop ambitieux sous peine de devoir renoncer à certaines fonctionnalités et de ne pas rendre le livrable annoncé initialement.

\begin{figure}[H]
    \centering
    \includegraphics[width=0.5\textwidth]{img/triangle_QCD.png}
    \caption{Triangle DQC}
\end{figure}

\subsection{Matrice SWOT}
Afin d’avoir une vision plus globale de nos ressources et des facteurs interne et externe agissant sur le projet, nous avons ensuite réalisé la matrice SWOT (Strengths, Weaknesses, Opportunities, Threats) de notre projet.

\begin{figure}[H]
    \centering
    \includegraphics[width=0.75\textwidth]{img/SWOT.png}
    \caption{Matrice SWOT}
\end{figure} 

On peut ainsi remarquer que notre projet présente de nombreux points fort notamment grâce aux connaissances acquises lors des cours de Télécom Nancy mais également de part l’expérience forte de deux des membres de l’équipe projet qui ont déjà réalisé des applications similaires.  Cependant, plusieurs facteurs internes constituent nos faiblesses notamment les courts délais qui nous oblige à être concis et efficaces dans notre travail, ou encore le faible bagage informatique de deux des membres de l’équipe. Néanmoins, ces lacunes constituent pour eux l’opportunité d’apprendre, et de progresser avec l’aide des membres expérimentés de l’équipe.

De plus, nous devons anticiper les charges de travail dans le cadre de notre formation à Télécom Nancy qui s'avèrent être plus élevée en décembre lors des partiels de fin d'année. Nous allons donc devoir prendre cela en compte dans notre gestion des tâches. 

\subsection{Profil de projet}
Afin d’avoir une vision plus globale sur notre projet, nous avons également réalisé le profil du projet (le budget étant égal à 0, nous avons choisi de ne pas le représenter dans notre profil). On remarque que, du fait des nombreuses fonctionnalités que nous avons l’intention d’implémenter dans notre application, que notre projet est de taille moyenne mais de complexité élevée.

Cependant, les enjeux du projet ne sont pas très importants (en dehors de la note finale qui compte dans notre moyenne) car l'échec du projet n'engendra pas la chute d'une organisation et le budget est négligeable.

De plus, au vu de l’état de l’art établi, l’innovation du projet est importante puisque nous avons choisi de combiner différentes fonctionnalités existantes de plusieurs applications et d’en rajouter de nouvelles.

\begin{figure}[H]
    \centering
    \includegraphics[width=0.75\textwidth]{img/profil_projet.png}
    \caption{Profil du projet}
\end{figure} 

\subsection{WBS~: comment concrétiser l’application}
Ceci étant fait, nous avons maintenant choisi de détailler les lots de travail à effectuer pour fabriquer notre application. Nous avons ainsi réalisé le WBS (Work Breakdown Structure) de notre application~: il apparait ainsi les grandes étapes de notre projet que sont~: definition du cadre de l’application, développement des fonctionnalités de l’application et écriture du rapport.
\begin{figure}[H]
    \centering
    \includegraphics[width=1\textwidth]{img/WBS.png}
    \caption{WBS}
\end{figure} 

\subsection{Diagramme de Gantt~: planification}
Maintenant que nous avons un détail des lots de travail qui constitue notre application, il faut maintenant les mettre en relation pour créer un planning efficace où chaque tâche est effectuée dans l’ordre.
\begin{figure}[H]
    \centering
    \includegraphics[width=1\textwidth]{img/gantt.png}
    \caption{Diagramme de GANTT}
\end{figure} 
Ce diagramme est une première version générale des tâches à effectuer, il sera modifié et détaillé davantage une fois la conception et les maquettes du projet réalisé.

\subsection{Matrice RACI}
Maintenant que toutes les étapes sont planifiées, nous devons répartir le travail entre les membres de l’équipe. On utilise ainsi une matrice RACI synthétisant les rôles de chacun.

\begin{figure}[H]
    \centering
    \includegraphics[width=1\textwidth]{img/RACI.png}
    \caption{Matrice RACI}
\end{figure} 

\subsection{Gestion des risques}
Nous avons également penser à prévoir une partie des risques pouvant se dresser sur notre route, les risques les plus classiques étant 
la gestion du temps et le manque de comprehension de certaines personnes de l'équipe
\begin{figure}[H]
    \centering
    \includegraphics[width=1\textwidth]{img/Plan_gestion_risque.png}
    \caption{Plan de gestion des risques}
\end{figure} 

\section{Conclusion}

\section{Annexes}
\includepdf[pages=1]{../../cr_reu/octobre/16/cr_16_octobre.pdf}
\includepdf[pages=1]{../../cr_reu/octobre/18/cr_18_octobre.pdf}
\includepdf[pages=1]{../../cr_reu/novembre/18/cr_18_novembre.pdf}
\includepdf[pages=1]{../../cr_reu/novembre/30/cr_30_novembre.pdf}
\includepdf[pages=1]{../../cr_reu/decembre/16/cr_16_decembre.pdf}
\includepdf[pages=1]{../../cr_reu/decembre/27/cr_27_decembre.pdf}
\end{document}